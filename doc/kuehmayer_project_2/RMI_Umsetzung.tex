\subsection{Umsetzung}
In diesem Programm wurden drei Objekte angelegt. Welche dem Server und dem Client bekannt sein müssen. Diese Objekte werden zwischen Server und Client synchronisiert und der Client kann Funktionen der Objekte am Server aufrufen. Zu diesen Objekten gehört der Car\_Builder, welcher als Grundobjekt dient. Mit diesem Car\_Builder, kann man ein Objekt, ein Car, erzeugen, welches im weiteren Verlauf an den Car\_Calculator übergeben wird. Das Car an sich kann nur angesehen werden und es gibt keine Möglichkeit, es zu verändern. Der Car\_Calculator benötigt dann noch ein paar zusätzliche Infos und kann, basierend auf diesen Informationen dann zwei Berechnungen ausführen. Diese Berechnungen wird der Client am Server durchführen. Um ein paar Attribute des Cars besser darstellen zu können, gibt es drei Enums welche die jeweiligen Typen besser benennen. Die drei Grundobjekte werden dann am Server in dem Object\_Storage verwaltet. Dieser führt die vom Client gesendeten Operationen aus und überprüft, ob alle Bedingungen gegeben sind, gegebenenfalls gibt er das erwartete Ergebnis oder eine Fehlermeldung zurück. Damit die Kommunikation über Streams in assio mit Protobuff funktioniert, gibt es den Namespace message\_utility, welcher dafür sorgt, dass die zu String serialisierten Protobuff Objekte über einen Stream übertragen werden können. Dies ist notwendig, da in den serialisierten Protobuff Objekte \textbackslash n vorkommt, welches zu auslesen des Streams am Server führen würde. Daher werden die serialisierten Protobuff Objekt mit der Methode to\_ascii codiert und dann beim Empfänger wieder mit der Methode from\_ascii decodiert.Damit der Benutzer über den Client mit dem Server kommunizieren kann und Objekte erstellen und Methoden ausführen kann, gibt es das Repl, welches die Eingaben des Benutzers liest, diese validiert und die dem entsprechend Methoden, Änderungen, Ausgaben oder Kommunikationen mit dem Server vornimmt. Um generelle Konfigurationen auf dem Server und auf dem Client vor nehmen zu können, gibt es den Namespace config. In diesem Namespace liegen zwei Structs und mehrere Konfigurationsfunktionen. Eines der beiden Structs ist das Server Struct. In diesem Struct werden die Informationen bezüglich der IP-Adresse, des Portes und ob der Server vom Client geschlossen werden darf, gespeichert. Die Information über die IP-Adresse und des Portes ist für den Client wichtig, damit er eine Verbindung aufbauen kann. Die Information über den Port und ob der Server vom Client geschlossen werden darf, hingegen ist für den Server wichtig. Das zweite Struct, Log\_Settings, legt sowohl am Client als auch am Server die Konfiguration des Loggers fest und gibt diese in einem lesbaren Format aus. Die zusätzlichen Funktionen in dem config Namespace dienen der allgemeinen Konfiguration. Sie erlaben es sowohl den Client als auch den Server über eine JSON- oder eine TOML-Datei zu konfigurieren und validieren diese beiden Dateiarten auch. Des Weiteren gibt es noch die Funktion, den Server zu starten, welche benötigt wird, wenn der Client angibt, dass er den Server mit starten will.

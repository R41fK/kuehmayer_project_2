\subsubsection{Message\_utilitys}
Damit der Server und der Client kommunizieren können und dabei auch Objekte versendet werden können, gibt es für alle Objekte, welche auch übertragen werden, eine Protobuff-Message, welche an den Request oder den Reply übergeben werden kann. Um diese Messages mittels asio übertragen zu können, werden sie zu Strings serialisiert. Da diese serialisierten Strings allerdings auch \textbackslash n und andere ASCII-Steuerzeichen beinhalten können, werde diese Strings codiert, um dann ein \textbackslash n anzuhängen, um sie zu versenden. Damit die codierte Message möglichst kompakt, aber die Erstellung auch möglichst schnell geht, werden in diesem serialisierten String jedes Zeichen in seinen entsprechenden ASCII-Code umgewandelt. Die Kompaktheit und die Geschwindigkeit sind notwendig, damit beim Senden und Empfangen eine kürzest mögliche Decodierung, also eine kürzest mögliche Verzögerung stattfindet.

\vspace{10mm}
\begin{addmargin}[-3em]{0em}
\begin{minted}{c++}
string message_utility::to_ascii(string data) {
    stringstream sstream;

    for(char ch : data) {
        sstream << " " << (int)ch;
    }
    sstream << "\n";

    return sstream.str();
}


string message_utility::from_ascii(string data) {
    
    data = pystring::lstrip(data);
    stringstream sstream{};
    string new_string{};
    string temp{};

    sstream << data;

    while (sstream >> temp) {
        new_string.push_back((char) stoi(temp));
    }
    
    return new_string;
}
\end{minted}
\end{addmargin}
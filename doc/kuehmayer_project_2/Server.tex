\subsection{Server}
Beim Starten des Servers werden zuerst alle Parameter auf ihre Richtigkeit überprüft und die Konfiguration basierend auf diesen Einstellungen vorgenommen. Wurde alles richtig konfiguriert, wird der Endpunkt des Servers erstellt. An diesen Endpunkt wird der eingetragenen Port übergeben. Ist dieser Port schon belegt, wirft der Endpunkt einen Error und das Programm beendet sich. Ist der Port jedoch nicht belegt, wird auf diesem Port auf anfragen gewartet. Dieses warten auf Anfragen passiert in einem eigenen Thread, sodass auch ein grpc-Server gestartet werden kann. Dieser Server überprüft auch zuerst, ob sein angegebener Port noch frei ist. Ist bei keinem der beiden Endpunkte ein Fehler aufgetreten, wird eine Nachricht ausgegeben, dass der Server gestartet ist. Jetzt wartet der Server auf Anfragen. Erhält der Server eine Anfrage, erstellt er für diese einen eigenen Thread und erstellt einen eigenen Object-Storage. Dieser überprüft jede Anfrage und sendet je nach dem eine Fehlermeldung oder ein Objekt zurück. Schließt der Client die Verbindung, wird der Object-Storage gelöscht und der Thread beendet sich. Der Server läuft solange weiter, bis er von einem Client oder durch eine Benutzereingabe das Zeichen bekommt, sich zu beenden. Erhält er dieses Zeichen, schließt er alle Verbindungen und beendet das Programm.
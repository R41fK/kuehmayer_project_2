\subsection{Client}
Beim Starten des Clients werden zuerst alle Parameter auf ihre Richtigkeit überprüft und die Konfiguration basierend auf diesen Einstellungen vorgenommen. Wurde alles richtig konfiguriert, startet der Client das Repl. Beim Start des Repls wird überprüft, ob sich der Client zu dem Server verbinden kann. Kann keine Verbindung aufgebaut werden, schließt sich der Client. Kann eine Verbindung aufgebaut werden, werden die Benutzereingaben überprüft und für jeden validen Befehl eine Funktion ausgeführt. Manche dieser Funktionen haben zur Folge, dass Objekte zwischen dem Server und dem Client synchronisiert werden müssen. Wird bei so einem Befehl festgestellt, dass der Server nicht mehr erreichbar ist, wird dem Benutzer die Option gegeben, das Programm zu beenden oder es erneut zu versuchen. Ist der Server jedoch erreichbar, werden die Objekte synchronisiert. Es gibt auch Befehle, welche es erfordern, dass der Server eine Methode ausführt und dem Client das Ergebnis zurückgibt. Will der Benutzer das Programm beenden, muss er den Endbefehl eingeben. Die Verbindung zum Server wird geschlossen und falls der Benutzer den Server mit dem Client mit gestartet hat, wird auch der Server geschlossen.

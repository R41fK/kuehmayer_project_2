\subsection{Integration-Tests}
Die Integration-Tests basieren ebenfalls auf der "doctest.h" Bibliothek, wodurch sie dieselbe Konfigurierbarkeit wie die Unit-Tests haben. Der einzige Unterschied zwischen diesen zwei Testmöglichkeiten ist, dass die Integration-Tests im Gegensatz zu den Unit-Tests einen eigenen Server starten, um die Funktion des Sendens und Empfangenes zu testen. Damit der Server gestartet werden kann, wird ein weiterer Prozess erstellt, welcher dann den Server startet. Dieser Server kann von einem Client herunter gefahren werden, denn mit dieser Technik wird der Server nach dem Erfolgen der Tests wieder geschlossen. Damit sicher gestellt wird, dass der Server auch läuft, wird vor den Tests die Verbindung überprüft. Diese Überprüfung bricht das Programm ab, wenn der Server nicht erreichbar ist. Kann der Server nicht gestartet werden, da schon ein Programm auf diesem Port lauscht und es das in diesem Projekt entwickelte Server-Programm ist, laufen die Integration-Tests trotzdem ab. Es kann nur zu Komplikationen führen, wenn der Server so konfiguriert ist, dass der Client ihn schließen kann, da die Integration-Tests den Server am Ende schließen.